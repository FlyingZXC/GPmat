%{
\begin{octave}
%}
sheffieldmlToolboxes;
%{
\end{octave}

\title{SheffieldML - MATLAB Software}

\begin{document}
\section{Examples}

\section{Gaussian Process Models}

\subsection{Functions from Gaussians}

This example shows how points which look like they come from a
function to be sampled from a Gaussian distribution. The sample is 25
dimensional and is from a Gaussian with a particular covariance.

\begin{octave}
%}
demGpSample
%{
\end{octave}

\begin{center}
\inputdiagram{gpSample}\hfill\inputdiagram{gpCovariance}

\emph{Left} A single, 25 dimensional, sample from a Gaussian
distribution. \emph{Right} the covariance matrix of the Gaussian
distribution.

\end{center}


\subsection{Joint Distribution over two Variables}

Gaussian processes are about conditioning a Gaussian distribution
on the training data to make the test predictions. To illustrate this
process, we can look at the joint distribution over two variables.

\begin{octave}
%}
 demGpCov2D([1 2])
%{
\end{octave}

Gives the joint distribution for $f_1$ and
$f_2$. The plots show the joint distributions as well
as the conditional for $f_2$ given
$f_1$.

\begin{center}
\inputdiagram{demGpCov2D1_2_3} 0.5\textwidth \inputdiagram{demGpCov2D1_5_3} 0.5\textwidth 

\emph{Left} Blue line is
contour of joint distribution over the variables $f_1$
and $f_2$. Green line indicates an observation of
$f_1$. Red line is conditional distribution of
$f_2$ given $f_1$. \emph{Right} Similar
for $f_1$ and $f_5$.  
\end{center}



\subsection{Different Samples from Gaussian Processes}

A script is provided which samples from a Gaussian process with the
provided covariance function.

\begin{octave}
%}
 gpSample('rbf', 10, [1 1], [-3 3], 1e5)
%{
\end{octave}

will give 10 samples from an RBF covariance function with a
parameter vector given by [1 1] (inverse width 1, variance 1) across
the range -3 to 3 on the $x$-axis. The random seed will be set to
1e5.

\begin{octave}
%}
 gpSample('rbf', 10, [16 1], [-3 3], 1e5)
%{
\end{octave}

is similar, but the inverse width is now set to 16 (length scale 0.25).

\begin{center}\inputdiagram{gpSampleRbfSamples10Seed100000InverseWidth1Variance1} 0.5\textwidth \inputdiagram{gpSampleRbfSamples10Seed100000InverseWidth16Variance1} 0.5\textwidth \emph{Left} samples from an RBF style covariance function
with length scale 1. \emph{Right} samples from an RBF style covariance
function with length scale 0.25.  
\end{center}

Other covariance functions can be sampled, an interesting one is
the MLP covariance which is non stationary and can produce point
symmetric functions,


\begin{octave}
%}
 gpSample('mlp', 10, [100 100 1], [-1 1], 1e5)
%{
\end{octave}

gives 10 samples from the MLP covariance function where the `bias
variance' is 100 (basis functions are centered around the origin
with standard deviation of 10) and the `weight variance' is
100.

\begin{octave}
%}
 gpSample('mlp', 10, [100 1e-16 1], [-1 1], 1e5)
%{
\end{octave}

gives 10 samples from the MLP covariance function where the `bias
variance' is approximately zero (basis functions are placed on
the origin) and the `weight variance' is 100.

\begin{center}\inputdiagram{gpSampleMlpSamples10Seed100000WeightVariance100BiasVariance100Variance1} 0.5\textwidth \inputdiagram{gpSampleMlpSamples10Seed100000WeightVariance100BiasVariance1e-16Variance1}
0.5\textwidth \emph{Left} samples from an MLP style covariance
function with bias and weight variances set to 100. \emph{Right}
samples from an MLP style covariance function with weight variance 100
and bias variance approximately zero.  
\end{center}


\subsection{Posterior Samples}

Gaussian processes are non-parametric models. They are specified by their covariance function and a mean function. When combined with data observations a posterior Gaussian process is induced. The demos below show samples from that posterior.

\begin{octave}
%}
  gpPosteriorSample('rbf', 5, [1 1], [-3 3], 1e5)
%{
\end{octave}

and 

\begin{octave}
%}
  gpPosteriorSample('rbf', 5, [16 1], [-3 3], 1e5)
%{
\end{octave}

\begin{center}\inputdiagram{gpPosteriorSampleRbfSamples5Seed100000InverseWidth1Variance1bw} 0.5\textwidth \inputdiagram{gpPosteriorSampleRbfSamples5Seed100000InverseWidth16Variance1bw} 0.5\textwidth 

\emph{Left} samples from the posterior induced by an RBF style covariance function
with length scale 1 and 5 `training' data points taken from a sine wave. \emph{Right} Similar but for a length scale of 0.25.  
\end{center}

\subsection{Simple Interpolation Demo}

This simple demonstration plots, consecutively, an increasing
number of data points, followed by an interpolated fit through the
data points using a Gaussian process. This is a noiseless system, and
the data is sampled from a GP with a known covariance function. The
curve is then recovered with minimal uncertainty after only nine data
points are included. The code is run with

\begin{octave}
%}
 demInterpolationGp
%{
\end{octave}

\begin{center}

\inputdiagram{demInterpolation3} 0.5\textwidth \inputdiagram{demInterpolation4} 0.5\textwidth

Gaussian process prediction \emph{left} after two points with a new
data point sampled \emph{right} after the new data point is included
in the prediction. 

\inputdiagram{demInterpolation7} 0.5\textwidth \inputdiagram{demInterpolation8} 0.5\textwidth

Gaussian process prediction \emph{left} after five points with a four
new data point sampled \emph{right} after all nine data points are
included. 
\end{center}

\subsection{Simple Regression Demo}

The regression demo very much follows the format of the
interpolation demo. Here the difference is that the data is sampled
with noise. Fitting a model with noise means that the regression will
not necessarily pass right through each data point.

The code is run with

\begin{octave}
%}
 demRegressionGp
%{
\end{octave}


\begin{center}
\inputdiagram{demRegression3} 0.5\textwidth \inputdiagram{demRegression4} 0.5\textwidth

Gaussian process prediction \emph{left} after two points with a new
data point sampled \emph{right} after the new data point is included
in the prediction. 

\inputdiagram{demRegression7} 0.5\textwidth \inputdiagram{demRegression8} width 0.5\textwidth

Gaussian process prediction \emph{left} after five points with a four
new data point sampled \emph{right} after all nine data points are
included. 
\end{center}

\subsection{Optimizing Hyper-parameters}

One of the advantages of Gaussian processes over pure kernel
interpretations of regression is the ability to select the hyper
parameters of the kernel automatically. The demo

\begin{octave}
%}
 demOptimiseGp
%{
\end{octave}

shows a series of plots of a Gaussian process with different length
scales fitted to six data points. For each plot there is a
corresponding plot of the log likelihood. The log likelihood peaks for
a length scale equal to 1. This was the length scale used to generate
the data.

\begin{center}\inputdiagram{demOptimiseGp1} 0.33\textwidth \inputdiagram{demOptimiseGp3} 0.33\textwidth \inputdiagram{demOptimiseGp5}
0.33\textwidth \inputdiagram{demOptimiseGp7} 0.33\textwidth \inputdiagram{demOptimiseGp9} 0.33\textwidth \inputdiagram{demOptimiseGp11} 0.33\textwidth \inputdiagram{demOptimiseGp13} 0.33\textwidth \inputdiagram{demOptimiseGp15} 0.33\textwidth \inputdiagram{demOptimiseGp17}
0.33\textwidth

From top left to bottom right, Gaussian process
regression applied to the data with an increasing length scale. The
length scales used were 0.05, 0.1, 0.25, 0.5, 1, 2, 4, 8 and
16. 

\inputdiagram{demOptimiseGp18} 0.5\textwidth 

Log-log plot of
the log likelihood of the data against the length scales. The log
likelihood is shown as a solid line. The log likelihood is made up of
a data fit term (the quadratic form) shown by a dashed line and a
complexity term (the log determinant) shown by a dotted line. The data
fit is larger for short length scales, the complexity is larger for
long length scales. The combination leads to a maximum around the true
length scale value of 1.

\end{center}

\subsection{Regression over Motion Capture Markers}

As a simple example of regression for real data we consider a motion capture data set. The data is <a href="http://accad.osu.edu/research/mocap/mocap_data.htm">from Ohio State University</a>. In the example script we perform Gaussian process regression with time as the input and the x,y,z position of the marker attached to the left ankle. To demonstrate the behavior of the model when the marker is lost, we remove data from This code can be run with

\begin{verbatim}  
demStickGp1 
\end{verbatim} 

\begin{octave}
%}
load('demStickGp1');
% Plot results
fillColor = [0.7 0.7 0.7];
tTest = linspace(0, 2, 200)';
[mu, varSigma] = gpPosteriorMeanVar(model, tTest);

for i = 1:length(outputIndex);
  figure
  fill([tTest; tTest(end:-1:1)], ...
       [mu(:, i); mu(end:-1:1, i)] ...
       + 2*[sqrt(varSigma(:, i)); -sqrt(varSigma(end:-1:1, i))], ...
       fillColor,'EdgeColor',fillColor)
  hold on;
  plot(t(testIndex), y(testIndex, outputIndex(i)), 'ko');
  b=plot(tTrain, yTrain(:, i), 'k.');
  a = plot(tTest, mu(:, i), 'k-');
  set(gca, 'xlim', [0 2])
  set(a, 'linewidth', 2);
  fileName = ['dem' capName 'Gp' num2str(experimentNo) 'Out' num2str(i)];
  printLatexPlot(fileName, '../tex/diagrams', 0.3*textWidth, options);
end
%{
\end{octave}

The code will optimize hyper parameters and show plots of the posterior process through the training data and the missing test points.

The result of the script is given in the plot below.  

\begin{center}
\inputdiagram{demStickGp1Out1} 0.3\textwidth \inputdiagram{demStickGp1Out2} 0.3\textwidth"> <\inputdiagram{demStickGp1Out3} width
="0.3\textwidth"> Gaussian process regression through the x (left), y (middle) and z (right) position of the left ankle. Training data is shown as black spots, test points removed to simulate a lost marker are shown as circles, posterior mean
prediction is shown as a black line and two standard deviations are
given as grey shading.\end{center}

Notice how the error bars are tight except in the region where the training data is missing and in the region where the training data disappears.

\subsection{Sparse Pseudo-input Gaussian Processes}

The sparse approximation used in this toolbox is based on the
Sparse Pseudo-input Gaussian Process model described by <a
href="http://ml.sheffield.ac.uk/~neil/publications/bibpage.cgi?keyName=Snelson:pseudo05&printAbstract=1">Snelson
and Ghahramani</a>. Also provided are the extensions suggested by <a
href="http://ml.sheffield.ac.uk/~neil/publications/bibpage.cgi?keyName=Quinonero:unifying05">Qui&ntilde;onero-Candela
and Rasmussen</a>. They provide a unifying terminology for describing
these approximations which we shall use in what follows.

There are three demos provided for Gaussian process regression in
1-D. They each use a different form of likelihood approximation. The
first demonstration uses the `projected latent variable'
approach first described by <a
href="http://ml.sheffield.ac.uk/~neil/publications/bibpage.cgi?keyName=Csato:sparse02&printAbstract=1">Csato
and Opper</a> and later used by <a
href="http://ml.sheffield.ac.uk/~neil/publications/bibpage.cgi?keyName=Seeger:fast03&printAbstract=1">Seeger
<i>et al.</i></a>. In the terminology of Qui&ntilde;onero-Candela and
Rasmussen (QR-terminology) this is known as the `deterministic
training conditional' (DTC) approximation.

To use this approximation the following script can be run.

\begin{octave}  
%}
demSpgp1dGp1 
%{
\end{octave} 

The result of the script is
given in the plot below.  

\begin{center}<\inputdiagram{demSpgp1dGp1} width
="0.5\textwidth"> Gaussian process using the DTC approximation with nine
inducing variables. Data is shown as black spots, posterior mean
prediction is shown as a black line and two standard deviations are
given as grey shading.\end{center}

The improved approximation suggested by Snelson and Ghahramani, in
QR-terminology this is known as the fully independent training
conditional (FITC). To try this approximation run the following script

\begin{octave}
%}  
demSpgp1dGp2 
%{
\end{octave}

The result of the script is given on the left of the plot below.

\begin{center}\inputdiagram{demSpgp1dGp2} width="0.49\textwidth"><img
src="demSpgp1dGp3} width="0.49\textwidth">

\emph{Left}: Gaussian process using the FITC approximation with nine
inducing variables. Data is shown as black spots, posterior mean
prediction is shown as a black line and two standard deviations are
given as grey shading. \emph{Right}: Similar but for the PITC
approximation, again with nine inducing variables.\end{center}

At the <a href="http://www.dcs.shef.ac.uk/ml/gprt/">Sheffield
Gaussian Process Round Table</a> Lehel Csato pointed out that the
Bayesian Committee Machine of <a
href="http://ml.sheffield.ac.uk/~neil/publications/bibpage.cgi?group=bcm&printAbstract=1">Schwaighofer
and Tresp</a> can also be viewed within the same framework. This idea
is formalised in <a
href="http://ml.sheffield.ac.uk/~neil/publications/bibpage.cgi?keyName=Quinonero:unifying05&printAbstract=1">Qui&ntilde;onero-Candela
and Rasmussen's</a> review. This approximation is known as the
`partially independent training conditional' (PITC) in
QR-terminology. To try this approximation run the following script

\begin{octave}
%}
 demSpgp1dGp3
%{
\end{octave}

The result of the script is given on the right of the plot above.

Finally we can compare these results to the result from the full
Gaussian process on the data with the correct hyper-parameters. To do
this the following script can be run.

\begin{octave}
%}
 demSpgp1dGp4
%{
\end{octave}

The result of the script is given in the plot below.

\begin{center}\inputdiagram{demSpgp1dGp4} width="0.5\textwidth"> Full Gaussian
process on the toy data with the correct hyper-parameters. Data is
shown as black spots, posterior mean prediction is shown as a black
line and two standard deviations are given as grey shaded
area.\end{center}

\section{Gaussian Process Latent Variable Model Examples}

The three approximations outlined above can be used to speed up learning in the GP-LVM. They have the advantage over the IVM approach taken in the <a href="http://ml.sheffield.ac.uk/~neil/gplvm/">original GP-LVM toolbox</a> that the algorithm is fully convergent and the final mapping from latent space to data space takes into account all of the data (not just the points in the active set).

As well as the new sparse approximation the new toolbox allows the GP-LVM to be run with dynamics as suggested by <a href="http://ml.sheffield.ac.uk/~neil/publications/bibpage.cgi?keyName=Wang:gpdm05&printAbstract=1">Wang <i>et al.</i></a>.

Finally, the new toolbox allows the incorporation of `back constraints' in learning. Back constraints force the latent points to be a smooth function of the data points. This means that points that are close in data space are constrained to be close in latent space. For the standard GP-LVM points close in latent space are constrained to be close in data space, but the converse is not true.

Various combinations of back constraints and different approximations are used in the exmaples below.

\subsection{Oil Data}

The `oil data' is commonly used as a bench mark for visualisation algorithms. For more details on the data see <a href="http://www.ncrg.aston.ac.uk/GTM/3PhaseData.html">this page</a>.

The <a href="http://ml.sheffield.ac.uk/~neil/gplvmcpp">C++ implementation of the GP-LVM</a> has details on training the full GP-LVM with this data set. Here we will consider the three different approximations outlined above.

\subsubsection{FITC Approximation}

In all the examples we give there will be 100 points in the active set. We first considered the FITC approximation. The script \texttt{demOilFgplvm1.m} runs the FITC approximation giving the result on the left of the figure shown below.

\begin{center}\inputdiagram{demOilFgplvm1} width="0.49\textwidth">\inputdiagram{demOilFgplvm2} width="49%">
\emph{Left}: GP-LVM on the oil data using the FITC approximation without back constraints. The phases of flow are shown as green circles, red crosses and blue plusses.  One hundred inducing variables are used. \emph{Right}: Similar but for a back-constrained GP-LVM, the back constraint is provided by a multi-layer perceptron with 15 hidden nodes.\end{center}

Back constraints can be added to each of these approximations. In the example on the right we used a back constraint given by a multi-layer perceptron with 15 hidden nodes. This example can be recreated with \texttt{demOilFgplvm2.m}.

\subsubsection{DTC Approximation}

The other approximations can also be used, in the figures below we give results from the DTC approximation. The can be recreated using \texttt{demOil3.m} and \texttt{demOil4.m}.

\begin{center}\inputdiagram{demOilFgplvm3} width="49%">\inputdiagram{demOilFgplvm4} width="49%">
\emph{Left}: GP-LVM on the oil data using the DTC approximation without back constraints. The phases of flow are shown as green circles, red crosses and blue plusses.  One hundred inducing variables are used. \emph{Right}: Similar but for a back-constrained GP-LVM, the back constraint is provided by a multi-layer perceptron with 15 hidden nodes.\end{center}

\subsubsection{PITC Approximation}

We also show results using the PITC approximation, these results can be recreated using the scripts \texttt{demOilFgplvm5.m} and \texttt{demOilFgplvm6.m}.

\begin{center}\inputdiagram{demOilFgplvm5} width="49%">\inputdiagram{demOilFgplvm6} width="49%">
\emph{Left}: GP-LVM on the oil data using the PITC approximation without back constraints. The phases of flow are shown as green circles, red crosses and blue plusses.  One hundred inducing variables are used. \emph{Right}: Similar but for a back-constrained GP-LVM, the back constraint is provided by a multi-layer perceptron with 15 hidden nodes.\end{center}

\subsubsection{Variational DTC Approximation}

Finally we also show results using the variational DTC approximation of Titsias, these results can be recreated using the scripts \texttt{demOilFgplvm7.m} and \texttt{demOilFgplvm8.m}.

\begin{center}\inputdiagram{demOilFgplvm7} width="49%">\inputdiagram{demOilFgplvm8} width="49%">
\emph{Left}: GP-LVM on the oil data using the variational DTC approximation without back constraints. The phases of flow are shown as green circles, red crosses and blue plusses.  One hundred inducing variables are used. \emph{Right}: Similar but for a back-constrained GP-LVM, the back constraint is provided by a multi-layer perceptron with 15 hidden nodes.\end{center}


\subsection{Back Constraints and Dynamics}

First we will demonstrate the dynamics functionality of the toolbox. We raw x-y-z values from a motion capture data set, the \texttt{Figure Run 1} example available <a href="http://accad.osu.edu/research/mocap/mocap_data.htm">from Ohio State University</a>. To run without dynamics use the script:
\texttt{
 demStickFgplvm1
\end{octave}

The results are given on the left of the figure below.

\begin{center}\inputdiagram{demStickFgplvm1} width="49%">
GP-LVM on the motion capture data without dynamics in the latent space. \end{center}

Notice that the sequence (which is a few strides of a man running) is split into several sub-sequences. These sub-sequences are aligned to the strides of the man. By introducing a dynamics prior, we can force the sequence to link up. Samples from the dynamics prior used are shown in the plot below.

\begin{center}\inputdiagram{dynamicsSamp1} width="49%">\inputdiagram{dynamicsSamp2} width="49%">
\inputdiagram{dynamicsSamp3} width="49%">\inputdiagram{dynamicsSamp4} width="49%">
Samples from the dynamics prior which is placed over the latent space. This prior has \emph{Left}: GP-LVM on the motion capture data without dynamics in the latent space. \emph{Right}: GP-LVM with dynamics. Samples from the dynamics prior used are given in the figure above.\end{center}

This prior is used in the model to obtain the results below,

\begin{octave}
 demStickFgplvm2
\end{octave}

\begin{center}\inputdiagram{demStickFgplvm2} width="49%">
GP-LVM with dynamics. Samples from the dynamics prior used are given in the figure above.\end{center}

Note now the circular form of the latent space. 

Back constraints can also be used to achieve a similar effect,

\begin{octave}
 demStickFgplvm3
\end{octave}

\begin{center}\inputdiagram{demStickFgplvm3} width="49%">
GP-LVM with back constraints. A RBF kernel mapping was used to form the back constraints with the inverse width set to 1e-4 (\emph{i.e.}length scale set to 100).\end{center}

\subsection{Loop Closure in Robotics}

In on-going work with Dieter Fox and Brian Ferris at the University of Washington we are interested in loop closure for robotic navigation, included as an example is a data set of a robot completing a loop while reading wireless access point signal strengths. To produce a neat track and close the loop it turns out it is necessary to use dynamics and back constraints as seen in the images below. These results can be recreated with \texttt{demRobotWireless1.m} through \texttt{demRobotWireless4.m}.
}

\begin{center}\inputdiagram{demRobotWireless1} width="49%">\inputdiagram{demRobotWireless2} width="49%">
\inputdiagram{demRobotWireless3} width="49%">\inputdiagram{demRobotWireless4} width="49%">
Use of back constraints and dynamics to obtain loop closure in a robot navigation example. \emph{Top Left}: GP-LVM without back constraints or dynamics, \emph{Top right}: GP-LVM with back constraints, no dynamics, \emph{Bottom Left}: GP-LVM with dynamics, no back constraints, \emph{Bottom right}: GP-LVM with back constraints and dynamics. \end{center}

\subsection{Vocal Joystick and Vowel Data}

Another ongoing piece of work with Jeff Bilmes and Jon Malkin involves embedding vowel sounds in a two dimensional space as part of <a href="http://ssli.ee.washington.edu/vj">vocal joystick</a> system. Jon has provided a simple data set of 2,700 examples of different vowels. These are embedded in a two dimensional latent space with and without back constraints.

\begin{center}\inputdiagram{demVowels2} width="49%">\inputdiagram{demVowels3} width="49%">
\emph{Left}: embedding of the vowel data without back constraints, \emph{Right}: embedding of the vowel data with back constraints. $/a/$ - red cross, $/ae/$ - green circle, $/ao/$ - blue plus, $/e/$ - cyan asterix, $/i/$ - magenta square, $/ibar/$ - yellow diamond, $/o/$ - red down triangle, $/schwa/$ - green up triangle, $/u/$ - blue left triangle.
\end{center}

\subsection{Larger Human Motion Data Sets}

For <a href="http://ml.sheffield.ac.uk/~neil/publications/bibpage.cgi?keyName=Lawrence:larger07&printAbstract=1">an AISTATS paper</a> we recreated an experiment from <a href="http://ml.sheffield.ac.uk/~neil/publications/bibpage.cgi?keyName=Taylor:motion06&printAbstract=1">Taylor <em>et al.</em>'s NIPS paper</a>. They created a data set from a motion capture data in the <a href="http://mocap.cs.cmu.edu">CMU data base</a> of running and walking. The data set can now be recreated using the <a href="/~neil/datasets/">DATASETS toolbox</a>. We repeated missing data experiments by Taylor et al.. The model learning for these experiments can be recreated with:

\begin{octave}
 demCmu35gplvm1
\end{octave}

for the four dimensional latent space, \texttt{demCmu35gplvm2} for the three dimensional latent space and \texttt{demCmu35gplvm3} for the five dimensional latent space. The test data reconstruction can then be performed for all models with \texttt{demCmu35gplvmReconstruct}. Taylor <i>et al.</i>'s nearest neighbour results can be recreated using \texttt{demCmu35TaylorNearestNeighbour}.

Data was pre-processed by mapping angles to be between -180 and 180 and scaling the data such that the variance of each dimension was one.
The quality of the trained model was evaluated using a missing data problem with a test sequence of data. The model was required to fill in either upper body angles or right leg angles. Results for the GP-LVM and nearest neighbour in both scaled space and original angle space are given in the table below.
\begin{center}
<table>
<tr>
<td></td><td align="center">Leg</td><td align="center">Leg</td><td align="center">Body</td><td align="center">Body</td>
</tr>
<tr>
<td></td><td align="center">Cumulative</td><td align="center">RMS</td><td align="center">Cumulative</td><td align="center">RMS</td>
</tr>
<tr>
<td></td><td align="center">Scaled</td><td align="center">Angles</td><td align="center">Scaled</td><td align="center">Angles</td>
</tr>
<tr>
<td>GP-LVM ($q$=3)</td><td align="right">11.4</td><td align="right">3.40</td><td align="right"><b>16.9</b></td><td align="right"><b>2.49</b></td>
</tr>
<tr>
<td>GP-LVM ($q$=4)</td><td align="right"><b>9.7</b></td><td align="right"><b>3.38</b></td><td align="right">20.7</td><td align="right">2.72</td>
</tr>
<tr>
<td>GP-LVM ($q$=5)</td><td align="right">13.4</b></td><td align="right">4.25</td><td align="right">23.4<td align="right">2.78</td>
</tr>
<tr>
<td>Scaled NN</td><td align="right">13.5</b></td><td align="right">4.44</td><td align="right">20.8<td align="right">2.62</td>
</tr>
<tr>
<td>Nearest Neighbour</td><td align="right">14.0</b></td><td align="right">4.11</td><td align="right">30.9<td align="right">3.20</td>
</tr>
</table>
\end{center}
The cumulative scaled error is a recreation of the error reported in Taylor <i>et al.</i> which was the average (across angles) cumulative sum (across time) of the squared errors in the down-scaled (\emph{i.e.} variance one) space of angles. We also present the root mean squared angle error for each joint which we find to be a little easier to interpret.

Taylor <i>et al.</i> used a slightly different representation of
the data set which included the absolute $x$ and $z$
position of the root node and rotation around the $y$-axis. For
this data set, this information does help, principally because the
subject seems to start in roughly the same position at the beginning
of each sequence. However, in general absolute position will not help,
so we discarded it in favour of a representation of these values in
terms of differences between frames. Finally Taylor <i>et al.</i>
concatenated two frames to form each data point for the model. We
chose not to do this as we wanted to test the ability of the Gaussian
process dynamics to fully recreate the data set. There results are given in their paper and summarised below.

\begin{center}
<table>
<tr>
<td></td><td align="center">Leg</td><td align="center">Body</td>
</tr>
<tr>
<td></td><td align="center">Cumulative</td><td align="center">Cumulative</td>
</tr>
<tr>
<td></td><td align="center">Scaled</td><td align="center">Scaled</td>
</tr>
<td>Binary Latent Variable Model</td><td align="right"><b>11.7</b></td><td align="right"><b>8.8</b></td>
</tr>
<tr>
<td>Scaled NN</td><td align="right">22.2</td><td align="right">20.5</td>
</tr>
</table>
\end{center}
Finally we show a plot of reconstructions of two of the angles in the data.

\begin{center}\inputdiagram{demCmu35gplvmLegReconstruct1_8} width="0.5\textwidth">\inputdiagram{demCmu35gplvmLegReconstruct1_9} width="0.5\textwidth">\end{center}
Prediction for first two angles of the right hip joint (see plots in <a href="http://ml.sheffield.ac.uk/~neil/publications/bibpage.cgi?keyName=Taylor:motion06&printAbstract=1">Taylor <i>et al.</i></a> for comparison). Dotted line is nearest neighour in scaled space, dashed line is GP-LVM with 4-D latent space.\end{center}



\section{Informative Vector Machines}

\section{\texttt{demClassificationOneIvm1}}

The first example given is \texttt{demClassificationOneIvm1} which is a simple classification data set, where only one direction of the input is relevant in determining the decision boundary. An ARD MLP kernel is used in combination with a linear kernel. The ARD parameters in the linear and MLP kernel are constrained to be the same by the line:

\begin{octave}

% Constrain the ARD parameters in the MLP and linear kernels to be the same.

model.kern = cmpndTieParameters(model.kern, {[4, 7], [5, 8]});

\end{octave}

The resulting classification is shown below.

\begin{center}\inputdiagram{demClassificationOneIvm1}

Decision boundary from the \texttt{demClassificationOneIvm1.m} example. Postive class is red circles, negative class green crosses and active points are yellow dots. Decision boundary shown in red, contours at 0.25 and 0.75 probability shown in blue.\end{center}




\section{\texttt{demClassificationTwoIvm1}}

The second example attempts to learn a Gaussian process give data that is sampled from a Gaussian process. The code is \texttt{demClassificationTwoIvm1}. The underlying Gaussian process is based on an RBF kernel with variance inverse width 10. The IVM learns an inverse width of 15 and gives the classification is shown below.

\begin{center}\inputdiagram{demClassificationTwoIvm1}

Decision boundary from the \texttt{demClassificationTwoIvm1.m} example. Postive class is red circles, negative class green crosses and active points are yellow dots. Decision boundary shown in red, contours at 0.25 and 0.75 probability shown in blue.\end{center}


\section{\texttt{demOrderedOneIvm1}}

In this example the ordered categorical noise model is used (ordinal regression). The data is a simple data set for which a linear one dimensional model suffices. The IVM is given a combination of an RBF and linear kernel with ARD.For the ordered categorical case there are several parameters associated with the noise model (in particular the category widths), these are learnt too. The model learns that the system is linear and only one direction is important. The resulting classification is given below.



\begin{center}\inputdiagram{demOrderedOneIvm1}

Decision boundary from the \texttt{demOrderedOneIvm1.m} example. Class 0 - red cross, Class 1 - green circles, Class 2 - blue crosses, Class 3 - cyan asterisks, Class 4 - pink squares, Class 5 - yellow diamonds. Class 6 - red triangles. Active points are yellow dots, note that because the kernel is linear by now the most informative points tend to be at the extrema. Decision boundaries shown in red, contours at 0.25 and 0.75 probability shown in blue.\end{center}



\section{\texttt{demOrderedTwoIvm1}}

Another example with the ordered categorical noise model, here the data is radial, the categories being along the radius of a circle. The IVM is given a combination of an RBF and linear kernel with ARD. Again there are several parameters associated with the noise model, and these are learnt using \texttt{ivmOptimiseNoise}. The resulting classification is given below.



\begin{center}\inputdiagram{demOrderedTwoIvm1}

Decision boundary from the \texttt{demOrderedTwoIvm1.m} example. Class 0 - red cross, Class 1 - green circles, Class 2 - blue crosses, Class 3 - cyan asterisks, Class 4 - pink squares, Class 5 - yellow diamonds. Class 6 - red triangles. Active points are yellow dots, note that because the kernel is linear by now the most informative points tend to be at the extrema. Decision boundaries shown in red, contours at 0.25 and 0.75 probability shown in blue.\end{center}





\section{\texttt{demRegressionOneIvm1}}

In this example the Gaussian noise model is used (standard regression). The data is sampled from a Gaussian process, only one input dimension is important. The IVM is given a combination of an RBF and linear kernel with ARD. The resulting regression is given below.



\begin{center}\inputdiagram{demRegressionOneIvm1}

Regression from the example \texttt{demRegressionOneIvm1.m}. Targets are red dots and active points are yellow dots.\end{center}



\section{\texttt{demRegressionTwoIvm1}}

A second example with Gaussian noise, sampled from a Gaussian process, but this time with differing length scales.



\begin{center}\inputdiagram{demRegressionTwoIvm1}

Regression from the example \texttt{demRegressionTwoIvm1.m}. Targets are red dots and active points are yellow dots.\end{center}



\section{Benchmark Data Sets}



The function \texttt{ivmGunnarData} allows you to test the IVM on Gunnar Raetsch's benchmark data sets. Download the data sets, <a href="http://ida.first.fraunhofer.de/projects/bench/benchmarks.htm">from here</a> and expand the ringnorm data set into '\$DATASETSDIRECTORY/gunnar/ringnorm'. Then run the following script.





\begin{octave}
%}
ivmGunnarData('ringnorm', 1, {'rbf', 'bias', 'white'}, 1, 100)
%{
\end{octave}
\begin{verbatim}

Final model:

IVM Model:

 Noise Model:

  Probit bias on process 1: 0.0439

  Probit Sigma2: 0.0000

 Kernel:

  Compound kernel:

    RBF inverse width: 0.0866 (length scale 3.3984)

    RBF variance: 1.2350

    Bias Variance: 8.2589

    White Noise Variance: 0.0000

Test Error 0.0183

Model likelihood -56.7120

\end{verbatim}



You can try any of the data sets by replacing ringnorm with the relevant data set (note that they don't all work with only 100 active points inas in the example above, for example the 'banana' data set needs 200 active points to get a reasonable result,



\begin{octave}

 ivmGunnarData('banana', 1, {'rbf', 'bias', 'white'}, 1, 200)
\end{octave}

\begin{verbatim}
Final model:

IVM Model:

 Noise Model:

  Probit bias on process 1: 0.1067

  Probit Sigma2: 0.0000

 Kernel:

  Compound kernel:

    RBF inverse width: 1.6411 (length scale 0.7806)

    RBF variance: 0.2438

    Bias Variance: 0.0000

    White Noise Variance: 0.0148

Test Error 0.1129

Model likelihood 175.3588

\end{verbatim}





\begin{center}
\inputdiagram{demBanana1}

Decision boundary from the banana example. Postive class is red circles, negative class green crosses and active points are yellow dots. Decision boundary shown in red, contours at 0.25 and 0.75 probability shown in blue.
\end{center}



\section{Null Category Noise Model}



\section{Examples}



The toy data example in the papers can be recreated using:



\begin{octave}

 demUnlabelledOneIvm1

\end{octave}



and leads to the decision boundary given below. A standard IVM based classifier can be run on the data using



\begin{octave}

 demUnlabelledOneIvm2

\end{octave}



\begin{center}\inputdiagram{demUnlabelledOneIvm1}\inputdiagram{demUnlabelledOneIvm2}

The null category noise model run on toy data. \emph{Top}: using the null category, the true nature of the decision boundary is recovered. \emph{Bottom}: the standard IVM, does not recover the true decision boundary.\end{center}



The other USPS digit classification example given in the NIPS paper can be re-run with:



\begin{octave}

 demThreeFiveIvm

\end{octave}



Be aware that this code can take some time to run. The results, in the form of averaged area under ROC curve against probability of missing label, can be plotted using





\begin{octave}
%}
 demThreeFiveResults
%{
\end{octave}

\begin{center}

\inputdiagram{demThreeFive} 

Plot of average area under ROC curve against probability of label being present. The red line is the standard IVM based classifier, the blue dotted line is the null category noise model based classifier, the green dash-dot line is the a normal SVM and the mauve dashed line is the transductive SVM.

\end{center}




\end{document}
%}